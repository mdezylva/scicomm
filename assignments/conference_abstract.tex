\documentclass[12pt]{extarticle}
\usepackage[utf8]{inputenc}
\usepackage[useregional]{datetime2}

\title{A new use for the universe's baby pictures: Using the oldest light in the sky to search for large scale structure}
\author{Mitchell de Zylva, 756539, Word Count: 286}
\date{\today}

\begin{document}

\maketitle

The afterglow of the Big Bang, known as the Cosmic Microwave Background (CMB) provides a powerful tool which can be used to probe the nature of the universe. From it, and from other independant observations, it is known that the Universe contains a certain proportion of regular matter, dark matter, and dark energy. However, at close astrophysical distances, even the regular matter that we expect to find is missing, invisible to optical observations, which throws doubt on very well established cosmological models. 

\par Simulations have proposed that approximately half of the matter in the universe is present in its large scale structure, in the filaments that exist between galaxies. The size of the universe proves to be an obstacle, since this material is so sparse that separating its signal from the much brighter starlight surrounding it, is virtually impossible.
 
\par One possible method for locating this missing matter is by searching for the incredibly weak signal that it produces in the CMB. Given we know where we might expect to find some of this missing matter, we can stack the signals from the CMB on top of each-other, revealing the location and amount of matter needed to confirm our models. 

\par Utilising a large scale galaxy survey, known as the Dark Energy Survey, we can select a type of galaxy which we know traces the large scale structure of the universe. We then combined this with a high precision ground based experiment, the South Pole Telescope, and obtained a signal  with less noise than other implementations of this method.
 
\par By measuring the missing baryon fraction we therefore confirm current cosmological models,  placing constraints on large scale structure formation, as well as providing valuable information to other areas of astrophysics.


\end{document}